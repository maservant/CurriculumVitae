\documentclass[11pt,a4paper]{article}
\usepackage[margin=0.9in]{geometry}
\usepackage[utf8]{inputenc}
\usepackage{mdwlist}
\usepackage[hidelinks]{hyperref}
\usepackage{tgpagella}
\usepackage[T1]{fontenc}
\usepackage{textcomp}
\pagestyle{empty}
\setlength{\tabcolsep}{0em}

% indentsection style, used for sections that aren't already in lists
% that need indentation to the level of all text in the document
\newenvironment{indentsection}[1]%
{\begin{list}{}%
	{\setlength{\leftmargin}{#1}}%
	\item[]%
}
{\end{list}}

% opposite of above; bump a section back toward the left margin
\newenvironment{unindentsection}[1]%
{\begin{list}{}%
	{\setlength{\leftmargin}{-0.5#1}}%
	\item[]%
}
{\end{list}}

% format two pieces of text, one left aligned and one right aligned
\newcommand{\headerrow}[2]
{\begin{tabular*}{\linewidth}{l@{\extracolsep{\fill}}r}
	#1 &
	#2 \\
\end{tabular*}}

\newcommand{\Cplusplus}
{C\nolinebreak[4]\hspace{-.05em}\raisebox{.22ex}{\footnotesize\bf ++}}

\begin{document}
{\Large \textbf{Marc-André Servant}}

256 rue de Carignan, Repentigny, Québec, Canada J5Y 4A9

+1-514-641-6011 \ \ -- \ \ \href{mailto:marc.andre3@hotmail.com}{marc.andre3@hotmail.com}
\\

\hrule
\vspace{-0.4em}
\subsection*{Expérience de travail}

\begin{itemize}
	\parskip=0.1em

	\item
	\headerrow
		{\textbf{TelcoBridges, Inc.}}
		{\textbf{Boucherville, Canada}}
	\\
	\headerrow
		{\emph{Stagiaire en conception de logiciel}}
		{\emph{2016}}
	\begin{itemize*}
		\item Conception de logiciels pour le projet de bloc d'alimentation pour serveurs de Telcodium
		\item Développement du micrologiciel s'exécutant sur le microcontrôleur embarqué
		(en C)
		\item Écriture de modules du micrologiciel pour permettre la communication avec le serveur hôte
		par les protocoles \textit{PMBus} et \textit{SMBus}
		\item Création d'une application permettant au serveur de communiquer avec le produit
		et d'afficher un diagnostic électrique (en C/\Cplusplus)
		\item Programmation d'une charge électrique ajustable et d'instruments de mesure pour effectuer
		des tests nocturnes automatisés de contrôle de qualité
	\end{itemize*}

\end{itemize}


\hrule
\vspace{-0.4em}
\subsection*{Éducation}

\begin{itemize}
	\parskip=0.1em

	\item 
	\headerrow
		{\textbf{École Polytechnique de Montréal}}
		{\textbf{Montréal, Canada}}
	\\
	\headerrow
		{\emph{Génie informatique}}
		{\emph{2015 -- présent}}
	\begin{itemize*}
		\item Moyenne universitaire de $3.82/4$
		\item Mention d'excellence au bulletin
	\end{itemize*}
	
	\item 
	\headerrow
		{\textbf{Cégep régional de Lanaudière à L'Assomption}}
		{\textbf{L'Assomption, Canada}}
	\\
	\headerrow
		{\emph{Diplôme d'études collégiales en Sciences de la nature}}
		{\emph{2013 -- 2015}}
	\begin{itemize*}
		\item Cote R de 32.3
	\end{itemize*}

\end{itemize}

\hrule
\vspace{-0.4em}
\subsection*{Projets}

\begin{itemize}
	\parskip=0.1em

	\item 
	\headerrow
		{\textbf{Enregistreur de données de vol pour micro fusée}}
		{\emph{2015 (Projet final du cégep)}}
	\begin{itemize*}
		\item Choix de composantes et construction d'un circuit électronique
		\item Collection des données de divers capteurs à l'aide d'un microcontrôleur
		\item Traitement des données du microcontrôleur sur ordinateur à l'aide d'un filtre de Kálmán
		\item Exportation des données d'altitude et d'accélération sur Excel
	\end{itemize*}

\end{itemize}


\hrule
\vspace{-0.4em}
\subsection*{Compétences}

\begin{indentsection}{\parindent}
\hyphenpenalty=1000
\begin{description*}
	\item[Expérience de travail sur systèmes embarqués avec des ressources limitées]
	\item[Langues parlées:]
	Français (langue maternelle), anglais (couramment)
	\item[Bonne connaissance:]
	C, \Cplusplus, HTML, CSS, JavaScript (incl. Node.js)
	\item[Connaissance de base:]
	Java, shell Bash, PHP, \LaTeX, Python
\end{description*}
\end{indentsection}

\end{document}
