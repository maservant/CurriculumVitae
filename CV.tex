\documentclass[11pt,a4paper]{article}
\usepackage[margin=0.9in]{geometry}
\usepackage[utf8]{inputenc}
\usepackage{mdwlist}
\usepackage[hidelinks]{hyperref}
\usepackage{tgpagella}
\usepackage[T1]{fontenc}
\usepackage{textcomp}
\pagestyle{empty}
\setlength{\tabcolsep}{0em}

% indentsection style, used for sections that aren't already in lists
% that need indentation to the level of all text in the document
\newenvironment{indentsection}[1]%
{\begin{list}{}%
	{\setlength{\leftmargin}{#1}}%
	\item[]%
}
{\end{list}}

% opposite of above; bump a section back toward the left margin
\newenvironment{unindentsection}[1]%
{\begin{list}{}%
	{\setlength{\leftmargin}{-0.5#1}}%
	\item[]%
}
{\end{list}}

% format two pieces of text, one left aligned and one right aligned
\newcommand{\headerrow}[2]
{\begin{tabular*}{\linewidth}{l@{\extracolsep{\fill}}r}
	#1 &
	#2 \\
\end{tabular*}}

\newcommand{\Cplusplus}
{C\nolinebreak[4]\hspace{-.05em}\raisebox{.22ex}{\footnotesize\bf ++}}

\begin{document}
{\Large \textbf{Marc-André Servant}}

29 rue Vincent, Repentigny, Québec, Canada J6A 4H9

(438) 630-1725 \ \ -- \ \ \href{mailto:marc@maservant.net}{marc@maservant.net}
\\

\hrule
\vspace{-0.4em}
\subsection*{Expérience de travail}

\begin{itemize}
	\parskip=0.1em

	\item
	\headerrow
		{\textbf{TelcoBridges, Inc.}}
		{\textbf{Boucherville, Québec}}
	\\
	\headerrow
		{\emph{Stagiaire en conception de logiciel}}
		{\emph{Été 2016, Été 2017}}
	\begin{itemize*}
		\item Développement de micrologiciel pour les projets d'onduleur et de bloc d'alimentation pour serveur de Telcodium.
		\item Développement du micrologiciel s'exécutant sur le microcontrôleur embarqué
		(en C)
		\item Expérimentation de différentes stratégies de contrôle de l'onduleur à l'aide d'équipement de
		laboratoire et d'un débogueur JTAG
		\item Développement d'un algorithme rapide de détection de fautes pour protéger le circuit avec des
		contraintes temps réel (en C avec des portions en assembleur)
		\item Écriture de modules du micrologiciel pour permettre la communication avec le serveur hôte
		par les protocoles \textit{PMBus} et \textit{SMBus}
		\item Création d'une application permettant au serveur de communiquer avec le produit
		et d'afficher un diagnostic électrique (en C/\Cplusplus)
		\item Programmation d'une charge électrique ajustable et d'instruments de mesure pour effectuer
		des tests nocturnes automatisés de contrôle de qualité
	\end{itemize*}

	\item
	\headerrow
		{\textbf{Ruiz Aerospace Manufacturing}}
		{\textbf{Laval, Québec}}
	\\
	\headerrow
		{\emph{Stagiaire en génie informatique}}
		{\emph{Été 2018}}
	\begin{itemize*}
		\item Développement de micrologiciel pour le TerviaHub, un outil d'infonuagique qui surveille la cuisson de la résine dans la fabrication de composites aérospatiaux
		\item Intégration d'une fonctionnalité de mise à jour sans fil dans le logiciel de démarrage du TerviaHub
		\item Programmation (en Python) et débogage sur le serveur en nuage pour implémenter une fonctionnalité de calibration assistée des capteurs de pression
		\item Implémentation d'une zone mémoire qui persiste au redémarrage pour y stocker des drapaux de fautes, permettant le diagnostic des plantages qui surviennent chez le client.
	\end{itemize*}

\end{itemize}


\hrule
\vspace{-0.4em}
\subsection*{Éducation}

\begin{itemize}
	\parskip=0.1em

	\item 
	\headerrow
		{\textbf{École Polytechnique de Montréal}}
		{\textbf{Montréal, Québec}}
	\\
	\headerrow
		{\emph{Génie informatique}}
		{\emph{2015 -- présent}}
	\begin{itemize*}
		\item Moyenne universitaire de $3.39/4$
	\end{itemize*}
	
	\item 
	\headerrow
		{\textbf{Cégep régional de Lanaudière à L'Assomption}}
		{\textbf{L'Assomption, Québec}}
	\\
	\headerrow
		{\emph{Diplôme d'études collégiales en Sciences de la nature}}
		{\emph{2013 -- 2015}}
	\begin{itemize*}
		\item Cote R de 32.3
	\end{itemize*}

\end{itemize}


\hrule
\vspace{-0.4em}
\subsection*{Compétences}

\begin{indentsection}{\parindent}
\hyphenpenalty=1000
\begin{description*}
	\item[Expérience de travail sur systèmes embarqués avec des ressources limitées]
	\item[Langues parlées:]
	Français (langue maternelle), anglais (couramment)
	\item[Bonne connaissance:]
	C, \Cplusplus, HTML, CSS, JavaScript (incl. Node.js)
	\item[Connaissance de base:]
	Java, shell Bash, PHP, \LaTeX, Python
\end{description*}
\end{indentsection}

\end{document}
