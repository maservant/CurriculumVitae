\documentclass[11pt,a4paper]{article}
\usepackage[margin=0.9in]{geometry}
\usepackage[utf8]{inputenc}
\usepackage[french]{babel}
\usepackage{mdwlist}
\usepackage[hidelinks]{hyperref}
\usepackage{tgpagella}
\usepackage[T1]{fontenc}
\usepackage{textcomp}
\pagestyle{empty}
\setlength{\tabcolsep}{0em}

% indentsection style, used for sections that aren't already in lists
% that need indentation to the level of all text in the document
\newenvironment{indentsection}[1]%
{\begin{list}{}%
	{\setlength{\leftmargin}{#1}}%
	\item[]%
}
{\end{list}}

% opposite of above; bump a section back toward the left margin
\newenvironment{unindentsection}[1]%
{\begin{list}{}%
	{\setlength{\leftmargin}{-0.5#1}}%
	\item[]%
}
{\end{list}}

% format two pieces of text, one left aligned and one right aligned
\newcommand{\headerrow}[2]
{\begin{tabular*}{\linewidth}{l@{\extracolsep{\fill}}r}
	#1 &
	#2 \\
\end{tabular*}}

\newcommand{\Cplusplus}
{C\nolinebreak[4]\hspace{-.05em}\raisebox{.22ex}{\footnotesize\bf ++}}

\begin{document}
{\Large \textbf{Marc-André Servant}}

1660 boul. Roland-Therrien, Longueuil, Québec, Canada J4J 4M1

(438) 630-1725 \ \ -- \ \ \href{mailto:marc@maservant.net}{marc@maservant.net}

\hspace*{\bigskipamount}
\hrule
\subsection*{Expérience de travail}

\begin{itemize}
	\parskip=0.1em

	\item
	\headerrow
		{\textbf{Telcodium, Inc.}}
		{\textbf{Boucherville, Québec}}
	\\
	\headerrow
		{\emph{Concepteur de logiciels embarqués}}
		{\emph{2020-présent}}
	\begin{itemize*}
		\item Développement de micrologiciel pour les projets d'onduleur et de génératrices portatives au lithium-ion.
		\item Développement du micrologiciel s'exécutant sur le microcontrôleur embarqué
		(en C)
		\item Expérimentation de différentes stratégies de contrôle de l'onduleur à l'aide d'équipement de
		laboratoire et d'un débogueur JTAG
		\item Développement d'un algorithme rapide de détection de fautes pour protéger le circuit avec des
		contraintes temps réel (en C avec des portions en assembleur)
		\item Écriture du micrologiciel pour un contrôleur d'afficheur e-Ink avec support pour la fonction de
		rafraîchissement rapide.
		\item Écriture du micrologiciel pour un de système de contrôle des batteries (BMS).
		\item Programmation d'une charge électrique ajustable et d'instruments de mesure pour effectuer
		des tests automatisés de contrôle de qualité.
	\end{itemize*}

	\item
	\headerrow
		{\textbf{Ruiz Aerospace Manufacturing}}
		{\textbf{Laval, Québec}}
	\\
	\headerrow
		{\emph{Stagiaire en génie informatique}}
		{\emph{Été 2018}}
	\begin{itemize*}
		\item Développement de micrologiciel pour le TerviaHub, un outil d'infonuagique qui surveille la cuisson de la résine dans la fabrication de composites aérospatiaux
		\item Intégration d'une fonctionnalité de mise à jour sans fil dans le logiciel de démarrage du TerviaHub
		\item Programmation (en Python) et débogage sur le serveur en nuage pour implémenter une fonctionnalité de calibration assistée des capteurs de pression
		\item Implémentation d'une zone mémoire qui persiste au redémarrage pour y stocker des indications de fautes, permettant le diagnostic des plantages qui surviennent chez le client.
	\end{itemize*}

\end{itemize}

\hspace*{\bigskipamount}
\hrule
\subsection*{Éducation}

\begin{itemize}
	\parskip=0.1em

	\item 
	\headerrow
		{\textbf{École Polytechnique de Montréal}}
		{\textbf{Montréal, Québec}}
	\\
	\headerrow
		{\emph{Génie informatique}}
		{\emph{2015 -- présent}}
	\begin{itemize*}
		\item 114 crédits (études mises en pauses durant la pandémie)
	\end{itemize*}
	
	\item 
	\headerrow
		{\textbf{Cégep régional de Lanaudière à L'Assomption}}
		{\textbf{L'Assomption, Québec}}
	\\
	\headerrow
		{\emph{DEC (diplôme pré-universitaire) en Sciences de la nature}}
		{\emph{2013 -- 2015}}
	\begin{itemize*}
		\item Diplôme d'études collégiales en Sciences de la nature
	\end{itemize*}

\end{itemize}

\hspace*{\bigskipamount}
\hrule
\subsection*{Compétences}

\begin{indentsection}{\parindent}
\hyphenpenalty=1000
\begin{description*}
	\item[Expérience de travail sur systèmes embarqués avec des ressources limitées]
	\item[Langues parlées:]
	Français (langue maternelle), anglais (couramment)
	\item[Bonne connaissance:]
	C, \Cplusplus, Python, JavaScript (incl. Node.js)
	\item[Connaissance de base:]
	Java, shell Bash, PHP, \LaTeX, HTML
\end{description*}
\end{indentsection}

\end{document}
